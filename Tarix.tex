\documentclass[a4paper,14pt]{article}

\usepackage[margin=1.25in]{geometry}
\usepackage[T2A,T1]{fontenc}
\usepackage[utf8]{inputenc}
\usepackage[azerbaijani]{babel}

\title{Azərbaycan Tarixi}
\author{Dos. Şahverdiyev Ələmdar}
\date{20.09.2020}

\begin{document}
\maketitle
\newpage
\tableofcontents
\newpage
\part*{Mövzu I. Azərbaycan tarixinin nəzəri metodları}
%-Azərbaycan ərazisi haqqında\\*
%-Azərbaycan adının mənşəyi haqqında\\*
%-Azərbaycan tarixinin öyrənilmə üsulları\\*
%-Azərbaycan tarixində dövrləşmə

\section{Azərbaycan ərazisi haqqında}
\qquad Xəzər dənizinin qərb sahili ilə Urmiya gölünün qərb sahili və Samur çayı ilə Bağrıdağ dağı (Cənubi  Azərbaycanda Talış dağ silsiləsi ilə kəsişən Baqro dağı nəzərdə tutulur) arasındakı ərazi ən qədim zamanlardan Azərbaycan adlanır. Şimali Azərbaycan Qafqazın cənub-şərq hissəsini əhatə edir. Şərqdə Xəzər dənizi, şimalda Dağıstan, şimal-qərbdə Gürcüstan, cənub-qərbdə Ermənistan, cənubda isə Cənubi Azərbaycanla həmsərhəddir.1919-cu il statistik məlumata görə şimali Azərbaycanın ərazisi 114 min kv. km olub. Bunun 97,3 min kv.km-i mübahisəsiz ərazi olub. 1920-ci il 28 aprel Azərbaycan Cümhuriyyəti Bolşevik Rusiyası tərəfindən işğal olunandan sonra qeyri azərbaycanlılardan ibarət olan Moskva hökuməti tərəfindən Azərbaycan torpaqları Ermənistan və Gürcüstana "hədiyyə" olunub. Qarayazı ilə Borçalı Gürcüstana, Qazax rayonunun bir hissəsi, Zəngəzur, Şərur-dərəlyəz, Göycə, İrəvan isə Ermənistana verilib. SSRİ-in tərkibində Şimali Azərbaycanın ərazisi 86.6 min kv.km olub. Şimali Azərbaycan Qafqazın ərazisinin üçdə birini, Cənubi Qafqazın isə üçdə iki hissəsini təşkil edir. 1987-ci ildən başlayaraq keçmiş sovet rəhbərliyinin hərbi-maliyyə dəstəyi nəticəsində Ermənistan tərəfindən Azərbaycanın 20\% torpağı işğal olunub: Dağlıq Qarabağ ərazisi(4400 kv.km), Laçın(1992, 17-18 may, 1835 kv.km), Kəlbəcər(1993, 2-3 aprel, 1936 kv.km), Ağdam(1993, 23 iyul, 1094 kv.km), Füzuli(1993-23 avqust,1386 kv.km), Cəbrayıl(1993, 23 avqust), Qubadlı(1993, 31 avqust, 802 kv.km), Zəngilan(1993, 29 oktyabr, 707 kv.km).

1828-ci il Türkmənçay (Təbriz yaxınlığlnda kənddir) müqaviləsi ilə (çar Rusiyası ilə Qacarlar İranı arasında bağlanıb) Azərbaycan iki dövlət arasında bölüşdürülüb. Cənubi Azərbaycan İranın tərkibinə verilib. Cənubi Azərbaycan İranın şimal-qərbində yerləşir. Şimalda Araz çayı, Talış dağları və Muğan düzü Azərbaycanı iki hissəyə bölür, qərbdə Türkiyə ilə, cənub-qərbdə İraqla, şərqdə Xəzər dənizi, şimal-qərbdə isə Ermənistan və Naxçıvan Muxtar Respublikası ilə həmsərhəddir. Ərazisi 170-200 min kv.km göstərilir. 1920-ci il statistikasında bütöv Azərbaycan 200.5 min kv.km, müasir mənbələrdə isə 250 min kv.km. göstərilir. Hal-hazırda Cənubi Azərbaycanın ərazisi 4 vilayətə bölünür: Şərqi Azərbaycan, Qərbi Azərbaycan, Ərdəbil və Zəncan. Cənubi Azərbaycan Ərdəbil yaylasında Savalan dağı(4821 m) və Təbrizdən cənubda Səhənd dağı(3722 m) əhatə olunub. Bütöv Azərbaycan şərq-qərb, şimal-cənub beynəlxalq dəmir və avtomobil yollarının üzərində yerləşir. Bütöv Azərbaycanın ərazisi dünyanın bir çox dövlətlərinin ərazisi ilə müqayisə oluna bilər.

\section{Azərbaycan adının mənşəyi haqqında}
\hspace*{0.5cm} Vətənmizin adının mənası və tarixi ərazisi haqqında doğru, obyektiv məlumat verilməsi elmi, tərbiyəvi və siyasi əhəmiyyət daşıyır. Bu baxımdan Azərbaycan sözünə tarixi-coğrafi anlamda aydınlıq gətirilməsi zəruridir.

2.1. Azərbaycan adı çox qədimdə aşağıdakı dillərdə bu adları daşıyırdı:
Qədim fars dilində: Adərbadaqan, Adərabadeqan, Adərpadeqan, Adərbaycan, Azərbadeqan, Azərbaycan;
- Pəhləvi dilində Atun-padeqan;
- qədim yunan dilində Atro-paten;
- gürcü dilində Adarbadaqan.
Azərbaycanın bu şəkillərdə olan adını təhlil edərkən söz iki hissəyə bölünür,
1- ci hissə adzər, ader, adər, azər, atro, atər, atun bütün bu sözlər od deməkdir.
2- ci hissə badqan, abadqan, padeqan, bayqan, patakan, paykan və s. abadlıq deməkdir. Belə olan halda Azərbaycan sözü od ilə abad olan ölkə mənasını verir. Bu yanaşmanı təsdiq edən fakt ən qədim zamandan bu günə qədər Xəzər dənizinin qərb sahili ilə Urmiya gölünün qərb sahilləri arasında ərazinin yanar od və yeraltı yanar qazla zəngin olmasıdır (məs.Yanar dağ və Suraxanı Atəşgah qoruqları).

2.2. Bəzi müəlliflər (Zərdüştlik dininin tədqiqatçıları) Azərbaycan adının atəşpərəstlərin baş məbədlərinin adından götürüldüyünü iddia edirlər. Bu məbədin adı Azerbadeqan olub, Təbriz şəhərində yerləşib. Məbədin adını təhlil edərkən iki hissədən ibarət olduğunu görürük: biri azər,digəri də badeqandır. Birinci hissə "od" deməkdir;
- ikinci əski fars dilində "baqa" kəlməsidir, "Tanrı" mənasın verir. Bu surətlə "azərbadeqan" kəlməsini "Od Tanrısının məbədi” mənasında anlamaq olar.

2.3 Atropatın adından götürüldüyü qeyd olunur. Bu yanaşma yunan, roma müəllifləri tərəfindən irəli sürülüb. Akademik Z.Bünyadov isə "Atropatena tarixinin oçerkləri" əsərində bu yanaşmaya münasibət bildirib. O qeyd edirki Atropat adında hakim olmayıb və Atropat atəşpərəslik dinində ali kahin rütbəsi olub.

2.4 Türk mənşəli addır. Bu yanaşma Monqol (Hülaki) dövlətinin başçısı Qazan xanın vəziri Fəzlulah Rəşidədinin "Oğuznamə" əsərində öz əksini tapıb.
\end{document}
